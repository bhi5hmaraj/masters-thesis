% Chapter 2

\chapter{Conclusion} % Main chapter title

\label{Chapter6} % For referencing the chapter elsewhere, use \ref{Chapter2} 


We have reviewed some of the algorithms used for sampling a uniform spanning tree. The main impetus for going into the details of \citet{harvey2016generating} is due to the fact that it uses the Sherman-Morrison-Woodbury identity for updating the laplacian pseudoinverse. Recently \citet{DBLP:journals/corr/abs-2004-12739} also used the same identity for showing that \textbf{REACHABILITY} problem is in \texttt{DynFO} + \texttt{Mod} $2(\leq, +, \times)$. Hence we explored the possibility of using the same framework for sampling spanning trees in the dynamic setting. But it turns out that there are a lot of subtleties in coming up with a proper formulation. 

%----------------------------------------------------------------------------------------


%----------------------------------------------------------------------------------------





