% Chapter 1

\chapter{Random Walk Approach} % Main chapter title

\label{Chapter3} % For referencing the chapter elsewhere, use \ref{Chapter1} 

\section{Aldous, Broder}

Aldous \cite{aldous1990random} and Broder \cite{63516} independently invented the following simple random walk based algortihm.


\begin{figure}[h!]
% \centering
% \scalebox{0.7}{
\begin{algorithm}[H]
 \KwIn{$G = (V,E)$}
 \KwOut{A random spanning tree}
 
 Choose a starting vertex $s$ arbitrarily
 
 $T_V \leftarrow \{s\}, T_E \leftarrow \emptyset$
 
 \While{$|T_V| < |V$} {
 
    $next =_{u.a.r} N(s)$
    
    \If{$next \not\in T_V$} {
        $T_V = T_V \cup \{next\}$
        
        $T_E = T_E \cup \{(s, next)\}$
    }
    
    $s = next$
 
 }
 
 \KwRet{$T = (T_V, T_E)$}

 
  \caption{Aldous-Broder Algorithm}
\end{algorithm}
%  }
\end{figure} 

\subsection{Running Time}


\textbf{Cover Time}

$cov_G(u) := $ The expected number of steps for a random walk starting at $u$ to visit all the vertices in $G$ \\


\textbf{Cover Time of $G$} 

$cov_G := \max_{u \in V_G} cov_G(u)$  


It is known that $cov_G = \mathcal{O}(|V|\ |E|) = \mathcal{O}(|V|^3)$


\subsection{An Example}

% \centering
\begin{figure}[h!]

\centering
\begin {tikzpicture}[auto ,node distance =3 cm and 3cm ,on grid ,
semithick ,
state/.style ={ circle  , 
draw,black , text=black , minimum width =1 cm}]
\node[state] (C){$C$};
\node[state, fill=white!50!green] (A) [above =of C] {$A$};
\node[state] (B) [above right =of C] {$B$};
\node[state] (D) [right =of C] {$D$};
\path (C) edge[very thick] (D);
\path (B) edge[very thick] (D);
\path (A) edge[very thick] (B);
\path (C) edge[very thick] (A);
\path (D) edge[very thick] (A);
\end{tikzpicture}
%  \caption{Iteration 1}
\end{figure}


\begin{figure}[h!]
\centering
\begin{subfigure}[b]{.5\textwidth}
%  \begin{figure}
\centering
\begin {tikzpicture}[auto ,node distance =3 cm and 3cm ,on grid ,
semithick ,
state/.style ={ circle  , 
draw,black , text=black , minimum width =1 cm}]
\node[state] (C){$C$};
\node[state, fill=white!50!red] (A) [above =of C] {$A$};
\node[state, fill=white!50!green] (B) [above right =of C] {$B$};
\node[state] (D) [right =of C] {$D$};
\path (C) edge[very thick] (D);
\path (B) edge[very thick] (D);
\path (A) edge[very thick,blue] (B);
\path (C) edge[very thick] (A);
\path (D) edge[very thick] (A);
\end{tikzpicture}
%  \caption{Iteration 2}
% \end{figure}
\end{subfigure}
% \end{figure}

% \begin{figure}
 \centering
\begin{subfigure}{.5\textwidth}
 \centering
\begin {tikzpicture}[auto ,node distance =3 cm and 3cm ,on grid ,
semithick ,
state/.style ={ circle  , 
draw,black , text=black , minimum width =1 cm}]
\node[state] (C){$C$};
\node[state, fill=white!50!green] (A) [above =of C] {$A$};
\node[state, fill=white!50!red] (B) [above right =of C] {$B$};
\node[state] (D) [right =of C] {$D$};
\path (C) edge[very thick] (D);
\path (B) edge[very thick] (D);
\path (A) edge[very thick,blue] (B);
\path (C) edge[very thick] (A);
\path (D) edge[very thick] (A);
\end{tikzpicture}
%  \caption{Iteration 3}
\end{subfigure}

\centering
\begin{subfigure}{.5\textwidth}
 \centering
\begin {tikzpicture}[auto ,node distance =3 cm and 3cm ,on grid ,
semithick ,
state/.style ={ circle  , 
draw,black , text=black , minimum width =1 cm}]
\node[state] (C){$C$};
\node[state, fill=white!50!red] (A) [above =of C] {$A$};
\node[state, fill=white!50!red] (B) [above right =of C] {$B$};
\node[state, fill=white!50!green] (D) [right =of C] {$D$};
\path (C) edge[very thick] (D);
\path (B) edge[very thick] (D);
\path (A) edge[very thick,blue] (B);
\path (C) edge[very thick] (A);
\path (D) edge[very thick, blue] (A);
\end{tikzpicture}
%  \caption{iteration 4}
\end{subfigure}


% \only<5>{
\centering
\begin{subfigure}{.5\textwidth}
\centering
\begin {tikzpicture}[auto ,node distance =3 cm and 3cm ,on grid ,
semithick ,
state/.style ={ circle  , 
draw,black , text=black , minimum width =1 cm}]
\node[state] (C){$C$};
\node[state, fill=white!50!red] (A) [above =of C] {$A$};
\node[state, fill=white!50!green] (B) [above right =of C] {$B$};
\node[state, fill=white!50!red] (D) [right =of C] {$D$};
\path (C) edge[very thick] (D);
\path (B) edge[very thick] (D);
\path (A) edge[very thick,blue] (B);
\path (C) edge[very thick] (A);
\path (D) edge[very thick, blue] (A);
\end{tikzpicture}
 
\end{subfigure}
% }

% \only<6>{
\centering
\begin{subfigure}{.5\textwidth}
\centering
\begin {tikzpicture}[auto ,node distance =3 cm and 3cm ,on grid ,
semithick ,
state/.style ={ circle  , 
draw,black , text=black , minimum width =1 cm}]
\node[state] (C){$C$};
\node[state, fill=white!50!red] (A) [above =of C] {$A$};
\node[state, fill=white!50!red] (B) [above right =of C] {$B$};
\node[state, fill=white!50!green] (D) [right =of C] {$D$};
\path (C) edge[very thick] (D);
\path (B) edge[very thick] (D);
\path (A) edge[very thick,blue] (B);
\path (C) edge[very thick] (A);
\path (D) edge[very thick, blue] (A);
\end{tikzpicture}
 
\end{subfigure}
% }

% \only<7>{
\begin{subfigure}{.5\textwidth}
\centering
\begin {tikzpicture}[auto ,node distance =3 cm and 3cm ,on grid ,
semithick ,
state/.style ={ circle  , 
draw,black , text=black , minimum width =1 cm}]
\node[state, fill=white!50!green] (C){$C$};
\node[state, fill=white!50!red] (A) [above =of C] {$A$};
\node[state, fill=white!50!red] (B) [above right =of C] {$B$};
\node[state, fill=white!50!red] (D) [right =of C] {$D$};
\path (C) edge[very thick, blue] (D);
\path (B) edge[very thick] (D);
\path (A) edge[very thick,blue] (B);
\path (C) edge[very thick] (A);
\path (D) edge[very thick, blue] (A);
\end{tikzpicture}
 

\end{subfigure}
\end{figure}

% \pagebreak
% 
% \section{Wilson's Algorithm}
% 
% \citet{10.1145/237814.237880} proposed a different variant of random walk based algorithm which runs faster than the cover time of the graph. 

%----------------------------------------------------------------------------------------


%----------------------------------------------------------------------------------------





