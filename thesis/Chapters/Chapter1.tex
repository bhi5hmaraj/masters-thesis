% Chapter 1

\chapter{Introduction} % Main chapter title

\label{Chapter1} % For referencing the chapter elsewhere, use \ref{Chapter1} 

%----------------------------------------------------------------------------------------

% Define some commands to keep the formatting separated from the content 
\newcommand{\keyword}[1]{\textbf{#1}}
\newcommand{\tabhead}[1]{\textbf{#1}}
\newcommand{\code}[1]{\texttt{#1}}
\newcommand{\file}[1]{\texttt{\bfseries#1}}
\newcommand{\option}[1]{\texttt{\itshape#1}}

%----------------------------------------------------------------------------------------

Spanning trees have been a central object of study in graph theory for a long time. \citet{doi:10.1002/andp.18471481202} established a linear algebraic relationship between spanning trees of graphs and determinants while studying electric networks. We are interested in the following problem.

\begin{IP}[Uniform random spanning tree]
 Given an undirected connected graph $G = (V, E)$, sample a spanning tree $T$ with probability $\frac{1}{|\mathcal{T}|}$ where $\mathcal{T}$ denotes the set of all spanning trees of $G$. 
\end{IP}



Sampling spanning trees happens to be a primitive used in various problems such as 

\begin{itemize}
\item Constructing expanders (\cite{10.5555/1496770.1496834}, \cite{doi:10.1137/120890971})
\item Approximation algorithms for the travelling salesman problem(\cite{6108216}, \cite{doi:10.1287/opre.2017.1603})
\item Graph Sparcification (\cite{DBLP:journals/corr/abs-1005-0265}, \cite{dolev2016random})
\item Analysis of network reliability (\cite{10.5555/535891},\cite{doi:10.1002/net.3230200303}, \cite{colbourn1988estimating})
\item Sequence shuffling problem in Bioinformatics (\cite{KANDEL1996171})

\end{itemize}

\citet{10.1007/3-540-51687-5_27} proposed a distributed algorithm for this problem. Recently \citet{a11040053} implemented some of the random walk algorithms and have compared their efficiencies. 


\subsection{Algorithms for sampling random spanning trees}
There has been a lot of progress in this problem for the past 40 years. These algorithms can be broadly classified into 3 categories. 

\subsubsection{Matrix Based }
These algorithms are based on Kirchoff Matrix Tree theorem and involve computing determinants of the laplacian matrix of the graph  to sample spanning trees. These notions would be made clear in later part of the thesis.

\begin{itemize}
\item Random Spanning Tree, \cite{GUENOCHE1983214}
\item Unranking and ranking spanning trees of a graph, \cite{COLBOURN1989271}
\item Generating random combinatorial objects, \cite{KULKARNI1990185}
\item Two Algorithms for Unranking Arborescences, \cite{COLBOURN1996268}
\item Generating random spanning trees via fast matrix multiplication, \cite{harvey2016generating}
\end{itemize}


\subsubsection{Random Walk Based}
These algorithms simulate variants of random walk on the input graph and consutructs a spanning tree from this simulation.
\begin{itemize}
\item Generating random spanning trees, \cite{63516}
\item The random walk construction of uniform spanning trees and uniform labelled trees, \cite{aldous1990random}
\item Generating random spanning trees more quickly than the cover time, \cite{10.1145/237814.237880}
\item How to Get a Perfectly Random Sample from a Generic Markov Chain and Generate a Random Spanning Tree of a Directed Graph, \cite{Propp1998HowTG}

\end{itemize}


\subsubsection{Approximation Algorithms}
These are more recent algorithms which sample a uniform spanning tree with a high probability. The main theme here is to employ tools from approximation algorithms. 
\begin{itemize}
\item Faster generation of random spanning trees, \cite{10.5555/1747597.1748019}
\item Sampling Random Spanning Trees Faster than Matrix Multiplication, \cite{10.1145/3055399.3055499}
\item An Almost-Linear Time Algorithm for Uniform Random Spanning Tree Generation, \cite{10.1145/3188745.3188852}
\end{itemize}

\subsection{Motivation}

The main motivation with which we started exploring this problem was to extend sampling random spanning trees to a dynamic setting. We wanted to sample a spanning tree faster than computing it from scratch. 

\subsection{Structure of the thesis}

First we define the basic ideas which would be used later. Then we give a brief summary of the algorithms of some of the papers listed above. Then we explain the \citet{harvey2016generating} algorithm in detail.



