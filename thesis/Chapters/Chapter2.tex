% Chapter 2

\chapter{Background} % Main chapter title
\newcommand{\Mset}[2]{\ensuremath{\mathbb{M}_{#1 \times #2}}}
\newcommand{\reff}[1][e]{\ensuremath{R_#1^{\text{eff}}}}
\label{Chapter2} % For referencing the chapter elsewhere, use \ref{Chapter2} 

%----------------------------------------------------------------------------------------
% \section{Markov Chains}
% 
% \subsection{Fundamental theorem of Markov chain}
% 
% \subsection{Markov chain tree theorem}

To make the exposition self contained I have chosen the results which are relevant to understanding the Harvey, Xu algorithm. Most of these results are taken from \citet{TCS-054}

\section{Preliminary Linear Algebra}

\begin{Bf}
 If $A$ is a $n \times n$ real symmetric matrix, then all it's eigenvalues are real. 
\end{Bf}

\begin{Bf}[Eigenvectors of different eigenvalues are orthogonal]
 Let $\lambda_i$ and $\lambda_j$ be two eigenvalues of a symmetric matrix $A$ and $\textbf{u}_i, \textbf{u}_j$ be it's corresponding eigenvectors. If $\lambda_i \neq \lambda_j$ then $\langle \textbf{u}_i, \textbf{u}_j \rangle = 0$
\end{Bf}

\begin{Bf}[Min-Max Characterizations of Eigenvalues]
 If $A$ is a $n \times n$ real symmetric matrix, then the largest eigenvalue of $A$ is 
 
 $$ \lambda_n(A) = \max_{v \in \mathbb{R}^n \setminus \{0\}} \frac{v^T A v}{v^Tv}$$
 
 $$ \lambda_1(A) = \min_{v \in \mathbb{R}^n \setminus \{0\}} \frac{v^T A v}{v^Tv}$$
\end{Bf}

% https://tex.stackexchange.com/questions/128508/a-new-line-inside-subscript

\begin{Bf}[Courant-Weyl-Fisher min-max principle]
 \[ \lambda_k(A) = \min_{\substack{v \in \mathbb{R}^n \setminus \{0\}\\ v^Tu_i = 0 \ \forall i \in \{1, \dots , k-1\}}} \frac{v^T A v}{v^Tv} \]
 
 \[ \lambda_k(A) = \max_{\substack{v \in \mathbb{R}^n \setminus \{0\}\\ v^Tu_i = 0 \ \forall i \in \{1, \dots , k-1\}}} \frac{v^T A v}{v^Tv} \]
\end{Bf}

\begin{Bf}[Positive Semi Definite]
 A matrix $A$ is said to be positive semi definite (PSD) if $\lambda_1(A) \geq 0$ . $A$ is said to be positive definite if $\lambda_1(A) > 0$
\end{Bf}


\section{Laplacian of a Graph}

\begin{Bd}[Laplacian]
 For an undirected unweighted graph $G = (V, E)$ the Laplacian $L_G$ is a $|V| \times |V|$ matrix defined as 
 
  \[
    L_G(i, j) = 
\begin{cases}
    -1& \text{if } (i, j) \ \in E\\
    deg(i)& \text{if } i = j\\
    0              & \text{otherwise}
\end{cases}
\]

It can also be seen that 
$$L_G = D - A$$

where $D$ is a diagonal matrix with diagonal entries as degree of the corresponding vertex. 

 \end{Bd}

 \begin{Bd}[Weighted Laplacian]
 For an undirected weighted graph $G = (V, E)$ and a weight function $\textbf{w}: E \rightarrow \mathbb{R}_{\geq 0}$ the Laplacian $L_G$ is a $|V| \times |V|$ matrix defined as 
 
  \[
    L_G(i, j) = 
\begin{cases}
    -w(i,j)& \text{if } (i, j) \ \in E\\
    \displaystyle\sum_{(i,v) \in E} w(i, v)& \text{if } i = j\\
    0              & \text{otherwise}
\end{cases}
\]

\end{Bd}

\subsection{Properties of Laplacian}

\begin{Bf}
 The Laplacian of a graph is PSD
\end{Bf}

\begin{Bf}
 ker($L$) = span(\textbf{1})
\end{Bf}

\begin{Bf}
 Let $L$ be the Laplacian of a graph $G= (V,E)$, then
 $$ \lambda_2(L) > 0 \iff G \text{ is connected}$$
\end{Bf}


\subsection{Kirchoff Matrix Tree Theorem}

\begin{Bf}
 The number of spanning trees in a graph $G$ is \textbf{det}($L_G[i]$) (for any $i$) sad where $L_G$ denotes the Laplacian of $G$ and $L_G[i]$ denotes the matrix with $i^{th}$ row and column removed. 
\end{Bf}


\section{Electric Networks}

\subsection{Unweighted Graph}

\subsubsection{Incidence Matrix} 

Given an undirected unweighted graph $G=(V,E)$ with arbitrary orientation of edges. Let $B \in  \Mset{n}{m}$ \footnote{I have used a transposed version compared to \citet{TCS-054} so that it's consistent with the notation used later} called the edge-vertex incidence matrix defined as 

\[
    B(i, e) = 
\begin{cases}
 1 & \text{if } i \text{ is tail of } e\\
 -1 & \text{if } i \text{ is head of } e \\
 0 & \text{otherwise}
\end{cases}
\]

\begin{Bf}
 For a graph $G$ with arbitrarily chosen incidence matrix $B$ and Laplacian $L$, 
 $$B \cdot B^T = L$$
\end{Bf}


Given an unweighted graph $G = (V,E)$ we associate a electrical network by replacing each edge with a resistor with resistance $1 \ \Omega$. A current source is introduced in each vertex, denoted as $\textbf{c}_{\text{ext}} \in \mathbb{R}^n$. This induces a voltage at each vertex and current at each edge. Let's denote it as $\textbf{v} \in \mathbb{R}^n, \textbf{i} \in \mathbb{R}^m$



Let $i_{xy}$ denote the current from vertex $x$ to $y$ for edge $(x,y) \in E$. And $v_x$ denote the potential at a vertex $x \in V$.

\subsubsection{Kirchoff's current law}

Kirchoff's current law states that the algebraic sum of current into any vertex equals zero. 

$$ B \cdot \textbf{i} = \textbf{c}_{\text{ext}} $$

\subsubsection{Ohm's law}
Ohm's law states that electric current through an edge is directly proportional to the potential different across an edge and inversely proportional to the resistance of the edge.

$$ i_{xy} = \frac{v_x - v_y}{r_{xy}} $$

Since the resistance in the unweighted graph is $1 \ \Omega$ we have 

$$ \textbf{i} = B^T \cdot \textbf{v} $$ 

Combining Ohm's law and Kirchoff's law we get 

$$ L \cdot \textbf{v} = \textbf{c}_{\text{ext}} $$ 

\subsubsection{Laplacian pseudoinverse}



\subsubsection{Effective Resistance}

Effective resistance across 2 vertices $x,y$ is the resistance between $x,y$ if we treat the graph as a single resistor connected between $x$ and $y$. 

In our case we are mainly interested in the effective resistance across an edge $e = (x,y) \in E$.

\begin{Bd}
 Effective Resistance is the potential difference across an edge $e = (x,y)$ when $1A$ is inducted at $x$ and taken out at $y$. It is denoted as \reff
\end{Bd}

So we have $\textbf{c}_{\text{ext}} = e_x - e_y$ where $e_i$ denotes a vector with 1 in the $i^{th}$ index and 0 elsewhere. By definition of effective resistance we have 

\begin{align*}
 \reff &= (e_x - e_y)^T \cdot \textbf{v} \\
 \reff &= (e_x - e_y)^T \cdot L^+  \cdot \textbf{c}_{\text{ext}} \\
 \reff &= (e_x - e_y)^T \cdot L^+  \cdot (e_x - e_y) 
\end{align*}


\subsection{Weighted Graph}

There are a few subtle changes which needs to be incorporated for the weighted graph setting. Suppose $G=(V,E)$ be a undirected weighted graph with weight function $\textbf{w}: E \rightarrow \mathbb{R}_{\geq 0}$. Now the electric network of $G$ coressponds has edges replaced by a resistor with resistance $r_e = 1 / \textbf{w}(e) , \forall e \in E$. The intuition is that lower weight for an edge in graph $G$ means it's barely there hence it coressponds to higher resistance. And having no edge coressponds to infinite resistance. 

For deriving the other relations let $W \in \Mset{m}{m}$ be a diagonal matrix such that $W(e,e) = \textbf{w}(e)$. Now $L = B \ W \ B^T$ and Ohm's law becomes $\textbf{i} = W \ B^T \ \textbf{v} $. As it can be seen the formula for effective resistance remains the same

\pagebreak

% https://tex.stackexchange.com/questions/57152/how-to-draw-graphs-in-latex

\subsection{An Example}
Consider the following weighted graph $G$ in \textbf{Figure 2.1}
\definecolor {processblue}{cmyk}{0,0,0,0}


% https://tex.stackexchange.com/questions/37581/latex-figures-side-by-side

\begin{figure}[h!]
\centering
\begin{subfigure}{.5\textwidth}
  \centering
%   \includegraphics[width=.4\linewidth]{image1}

\begin {tikzpicture}[auto ,node distance =4 cm and 5cm ,on grid ,
semithick ,
state/.style ={ circle ,top color =white , bottom color = processblue!20 ,
draw,black , text=black , minimum width =1 cm}]
\node[state] (C){$A$};
\node[state] (A) [above =of C] {$D$};
\node[state] (B) [above right =of C] {$C$};
\node[state] (D) [right =of C] {$B$};
\path (C) edge node[below] {$1$} (D);
\path (B) edge node[right] {$5$} (D);
\path (A) edge node[above] {$10$} (B);
\path (C) edge node[left] {$4$} (A);
\end{tikzpicture}

  \caption{The original graph $G$}
  \label{fig:sub1}
\end{subfigure}%
\begin{subfigure}{.5\textwidth}
  \centering
%   \includegraphics[width=.4\linewidth]{image1}

\begin{circuitikz}[american]
 \put(0,0){0};
 \draw (0, -2) to[short, -*, i=$1 A$] (0,0) node[left]{$A$};
 \draw (0,0) to[R, l=\mbox{$1 \  \Omega$}] (3,0) node[right]{$B$};
 \draw (0,0) to[R, l=$0.25 \ \Omega$] (0,3) node[left]{$D$};
 \draw (3,0) to[R, l=\mbox{$0.2 \ \Omega$}] (3,3) node[right]{$C$};
 % this works, but it has wrong spacing
 \draw (0,3) to[R, l=$0.1 \ \Omega$] (3,3);
 \draw (3, 0) to[short, *-, i=$1 A$] (3,-2);
 \draw (3, -2) to[isource, l=$1 A$] (0, -2);
 \end{circuitikz}

\caption{The electric network version of $G$}
  \label{fig:sub2}
\end{subfigure}
\caption{An example of a graph and it's corresponding electric network}
\label{fig:test}
\end{figure}

Following the convention such that current going inside a vertex as negative and out as positive.

By Ohm's law we have 

\begin{align*}
i_{AD} &= 4  \, (v_A - v_D) \\ 
i_{DC} &= 10 \, (v_D - v_C) \\
i_{CB} &= 5  \, (v_C - v_B) \\
i_{AB} &= 1  \, (v_A - v_B) 
\end{align*}

By Kirchoff's current law we have 
\begin{align*}
 i_{AD} + i_{AB} &= 1 \\
 -i_{AD} + i_{DC} &= 0 \\
 -i_{DC} + i_{CB} &= 0 \\
 -i_{AB} - i_{CB} &= -1
\end{align*}

Now combining these two we get 

\begin{align*}
 5v_A  - 1 v_B - 0 v_C - 1 v_D &= 1 \\
 -1v_A + 6 v_B - 5 v_C - 0 v_D &= -1 \\
 0v_A - 5v_B + 15 v_C - 10 v_D &= 0 \\
 -4v_A - 0v_B - 10v_C + 14v_D &= 0
\end{align*}

And this is exactly what we would have gotten using $L \cdot \textbf{v} = \textbf{c}_{\text{ext}}$

%----------------------------------------------------------------------------------------





