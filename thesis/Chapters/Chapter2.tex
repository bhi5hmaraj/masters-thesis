% Chapter 2

\chapter{Background} % Main chapter title

\label{Chapter2} % For referencing the chapter elsewhere, use \ref{Chapter2} 

%----------------------------------------------------------------------------------------
% \section{Markov Chains}
% 
% \subsection{Fundamental theorem of Markov chain}
% 
% \subsection{Markov chain tree theorem}

To make the exposition self contained I have chosen the results which are relevant to understanding the Harvey, Xu algorithm. Most of these results are taken from \citet{TCS-054}

\section{Preliminary Linear Algebra}

\begin{Bf}
 If $A$ is a $n \times n$ real symmetric matrix, then all it's eigenvalues are real. 
\end{Bf}

\begin{Bf}[Eigenvectors of different eigenvalues are orthogonal]
 Let $\lambda_i$ and $\lambda_j$ be two eigenvalues of a symmetric matrix $A$ and $\textbf{u}_i, \textbf{u}_j$ be it's corresponding eigenvectors. If $\lambda_i \neq \lambda_j$ then $\langle \textbf{u}_i, \textbf{u}_j \rangle = 0$
\end{Bf}

\begin{Bf}[Min-Max Characterizations of Eigenvalues]
 If $A$ is a $n \times n$ real symmetric matrix, then the largest eigenvalue of $A$ is 
 
 $$ \lambda_n(A) = \max_{v \in \mathbb{R}^n \setminus \{0\}} \frac{v^T A v}{v^Tv}$$
 
 $$ \lambda_1(A) = \min_{v \in \mathbb{R}^n \setminus \{0\}} \frac{v^T A v}{v^Tv}$$
\end{Bf}

% https://tex.stackexchange.com/questions/128508/a-new-line-inside-subscript

\begin{Bf}[Courant-Weyl-Fisher min-max principle]
 \[ \lambda_k(A) = \min_{\substack{v \in \mathbb{R}^n \setminus \{0\}\\ v^Tu_i = 0 \ \forall i \in \{1, \dots , k-1\}}} \frac{v^T A v}{v^Tv} \]
 
 \[ \lambda_k(A) = \max_{\substack{v \in \mathbb{R}^n \setminus \{0\}\\ v^Tu_i = 0 \ \forall i \in \{1, \dots , k-1\}}} \frac{v^T A v}{v^Tv} \]
\end{Bf}

\begin{Bf}[Positive Semi Definite]
 A matrix $A$ is said to be positive semi definite (PSD) if $\lambda_1(A) \geq 0$ . $A$ is said to be positive definite if $\lambda_1(A) > 0$
\end{Bf}


\section{Laplacian of a Graph}

\begin{Bd}[Laplacian]
 For an undirected unweighted graph $G = (V, E)$ the Laplacian $L_G$ is a $|V| \times |V|$ matrix defined as 
 
  \[
    L_G(i, j) = 
\begin{cases}
    -1& \text{if } (i, j) \ \in E\\
    deg(i)& \text{if } i = j\\
    0              & \text{otherwise}
\end{cases}
\]

It can also be seen that 
$$L_G = D - A$$

where $D$ is a diagonal matrix with diagonal entries as degree of the corresponding vertex. 

 \end{Bd}

 \begin{Bd}[Weighted Laplacian]
 For an undirected weighted graph $G = (V, E)$ and a weight function $\textbf{w}: E \rightarrow \mathbb{R}_{\geq 0}$ the Laplacian $L_G$ is a $|V| \times |V|$ matrix defined as 
 
  \[
    L_G(i, j) = 
\begin{cases}
    -w(i,j)& \text{if } (i, j) \ \in E\\
    \displaystyle\sum_{(i,v) \in E} w(i, v)& \text{if } i = j\\
    0              & \text{otherwise}
\end{cases}
\]

\end{Bd}

\begin{Bf}
 The Laplacian of a graph is PSD
\end{Bf}

\begin{Bf}
 Let $L$ be the Laplacian of a graph $G= (V,E)$, then
 $$ \lambda_2(L) > 0 \iff G \text{ is connected}$$
\end{Bf}


\subsection{Kirchoff Matrix Tree Theorem}



\section{Electric Networks}

\subsubsection{Incidence Matrix} 
\footnote{I have used a transposed version compared to \citet{TCS-054} so that it's consistent with the notation used later}
\subsubsection{Unweighted Case}

Given an unweighted graph $G$ we associate a electrical network by replacing each edge with a resistor with resistance $1 \ \Omega$. 

% https://tex.stackexchange.com/questions/57152/how-to-draw-graphs-in-latex

\definecolor {processblue}{cmyk}{0,0,0,0}


% https://tex.stackexchange.com/questions/37581/latex-figures-side-by-side

\begin{figure}[h!]
\centering
\begin{subfigure}{.5\textwidth}
  \centering
%   \includegraphics[width=.4\linewidth]{image1}

\begin {tikzpicture}[auto ,node distance =4 cm and 5cm ,on grid ,
semithick ,
state/.style ={ circle ,top color =white , bottom color = processblue!20 ,
draw,black , text=black , minimum width =1 cm}]
\node[state] (C){$A$};
\node[state] (A) [above =of C] {$D$};
\node[state] (B) [above right =of C] {$C$};
\node[state] (D) [right =of C] {$B$};
\path (C) edge node[below] {$1$} (D);
\path (B) edge node[right] {$5$} (D);
\path (A) edge node[above] {$10$} (B);
\path (C) edge node[left] {$4$} (A);
\end{tikzpicture}

  \caption{The original graph $G$}
  \label{fig:sub1}
\end{subfigure}%
\begin{subfigure}{.5\textwidth}
  \centering
%   \includegraphics[width=.4\linewidth]{image1}

\begin{circuitikz}[american]
 \put(0,0){0};
 \draw (0, -2) to[short, -*, i=$1 A$] (0,0) node[left]{$A$};
 \draw (0,0) to[R, l=\mbox{$1 \  \Omega$}] (3,0) node[right]{$B$};
 \draw (0,0) to[R, l=$0.25 \ \Omega$] (0,3) node[left]{$D$};
 \draw (3,0) to[R, l=\mbox{$0.2 \ \Omega$}] (3,3) node[right]{$C$};
 % this works, but it has wrong spacing
 \draw (0,3) to[R, l=$0.1 \ \Omega$] (3,3);
 \draw (3, 0) to[short, *-, i=$1 A$] (3,-2);
 \draw (3, -2) to[isource, l=$1 A$] (0, -2);
 \end{circuitikz}

\caption{The electric network version of $G$}
  \label{fig:sub2}
\end{subfigure}
\caption{An example of a graph and it's corresponding electric network}
\label{fig:test}
\end{figure}


%----------------------------------------------------------------------------------------





